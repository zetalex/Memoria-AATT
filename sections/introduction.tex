
\renewcommand{\chaptername}{Sección}
\chapter{Introducción}

Esta memoria cubre todos los aspectos trabajados durante la Práctica 3 \cite{Guion} de la asignatura Fundamentos de VLSI. En ella, se incluye el proceso seguido para diseñar un Circuito de Lógica CMOS \cite{MOSFET} Dinámica y la simulación del esquemático que lo representa para responder a las cuestiones que se plantean en el guion de la práctica. La memoria se divide en las siguientes partes:

\begin{enumerate}
    \item \textbf{Síntesis de un Circuito CMOS Dinámico:} Se detalla el funcionamiento del circuito, las consideraciones a tener en cuenta a la hora del diseño, el esquemático del circuito y varios análisis transitorios, algunos de ellos paramétricos para hallar el dimensionado final del mismo. Este apartado se desarrollará en base a las 4 cuestiones que se plantean en la sección 2.2 del guion \cite{Guion}.
    \item \textbf{Encadenado de etapas dinámicas}: Incluye un esquemático en el que se encadenan y estimulan adecuadamente dos etapas del circuito CMOS dinámico diseñado previamente y un análisis transitorio del mismo. Este apartado se usará también para comentar por qué se ha de seguir un procedimiento específico a la hora de encadenar este tipo de circuitos.
    \item \textbf{Conclusiones:} Apartado donde se exponen de forma ordenada, las conclusiones a las que se ha llegado durante la realización de la práctica así como comentarios adicionales que merezca la pena realizar.
\end{enumerate}

Prueba de escritura GitHub

\begin{mylisting}[hbox,enhanced,drop shadow]{Java Program}
class MyClass
public static void main
\end{mylisting}
