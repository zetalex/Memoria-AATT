\pdfbookmark[0]{English title page}{label:titlepage_en}
\pagenumbering{roman} %use roman page numbering in the frontmatter
\aautitlepage{%
  \englishprojectinfo{
  Análisis y Dimensionado de un Circuito CMOS Dinámico %title
  }{%
    Diseño y simulación de Esquemáticos en Entorno Cadence %theme
  }{%
    2º cuatrimestre, 3er año del GTSIT %project period
  }{%
    GTE3B % project group
  }{%
    %list of group members
    Alejandro Gómez Gambín\\
    Pablo Martínez Sánchez
    
  }{%
    %list of supervisors
    Miguel Ángel Larrea Torres
  }{%
    1 % number of printed copies
  }{%
    \today % date of completion
  }%
}{%department and address
  \textbf{Fundamentos de VLSI}\\
  Universidad Politécnica de Valencia\\
  \href{http://www.upv.es}{http://www.upv.es}
}
{\begin{itemize}
    \item Profundizar en el diseño jerárquico y en la captura de esquemas digitales en el entorno de trabajo Cadence.
    \item Comprender el funcionamiento de los circuitos combinacionales dinámicos.
    \item Comprender el dimensionado de transistores asociado al del reloj y la disposición en cascada de etapas y secciones dinámicas.
    \item Entender el funcionamiento de la Lógica Dinámica N y el por qué de la conexión entre etapas.
\end{itemize}
}

